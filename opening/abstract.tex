\thispagestyle{plain}

\begin{center}
  \large
  \textbf{Resumen}
\end{center}

En la presente tesis se busca diseñar un guante que trackea el movimiento de las manos y reconocer los gestos que realizamos con el objetivo de poder traducir los movimiento y gestos a palabras en texto para asi tener la capacidad de traducir lenguaje de señas a texto o voz. Esto ayudaría a las personas con discapacidad auditiva a comunicarse de manera más efectiva con personas que no entienden el lenguaje de signos. Para ello se hará uso de una pequeña red neuronal que se ejecutará dentro de un pequeño microcontrolador que realizará la inferencia de la sucesión de signos y gestos que realiza la persona. En esta tesis se encontará el desarrollo completo de esta aplicación desde el diseño del hardware para el sensado, la recolección de datos, el entrenamiento de la red neuronal de clasificación y la programación del microcontrolador para ejecurar el modelo. Terminando así con un sistema embebido portátil capaz de traducir de Lenguaje de Signos Español al idioma Español hablado.
\\
\\
\textbf{Palabras clave:} inferencia, embebido, periferico, microcontrolador, entrenamiento, etiqueta, capa neuronal densa, sobreentrenamiento.

\newpage
\thispagestyle{plain}
\begin{center}
  \large
  \textbf{Abstract}
\end{center}

In this thesis, the aim is to design a glove that tracks hand movements and recognizes gestures, with the objective of translating these movements and gestures into words in text form. This capability would enable the translation of sign language into text or voice, thus helping hearing-impaired individuals to communicate more effectively with people who do not understand sign language. To achieve this, a small neural network will be employed, running on a microcontroller, to perform inference on the sequence of signs and gestures made by the user. This thesis will cover the complete development of this application, including the hardware design for sensing, data collection, training the classification neural network, and programming the microcontroller to execute the model. The result will be a portable embedded system capable of translating Spanish Sign Language into spoken Spanish.
\\
\\
\textbf{Keywords:} inference, embedded, edge device, microcontroller, training, label, fully connected layers, overfitting.
