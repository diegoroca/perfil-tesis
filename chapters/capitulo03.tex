El marco tenorico necesario para abordar la investicación se basa en los siguientes conceptos

\begin{itemize}

  \item \textbf{Energía y Radiación Solar}

  La energía solar es una fuente renovable obtenida directamente del Sol mediante la captación de su radiación electromagnética. Esta radiación, compuesta por radiación directa, difusa y global, varía según factores como la latitud, la altitud y las condiciones climáticas de una región. En el caso del distrito donde opera el comercio Huascarán, la alta disponibilidad de radiación solar lo convierte en un lugar ideal para la implementación de sistemas fotovoltaicos, aprovechando al máximo esta fuente energética limpia y abundante.


  \item \textbf{Sistemas Fotovoltáicos Aislados}

  Los sistemas fotovoltaicos aislados son soluciones energéticas diseñadas para operar de manera independiente a la red eléctrica convencional. En este caso el sistema actuará como una fuente alternativa cuando la red presente algun corte. Estos sistemas utilizan paneles solares para captar energía, que luego es almacenada en baterías para garantizar un suministro constante. Su implementación es especialmente útil en áreas donde el acceso a la red es inestable, como en el caso del comercio Huascarán, proporcionando una fuente confiable y autónoma de energía.


  \item \textbf{Almacenamiento de energía}

  El almacenamiento de energía es una parte esencial de los sistemas fotovoltáicos aislados, ya que permite acumular la energía generada durante el día para su uso en horas nocturnas o en momentos de baja radiación solar. Actualmente la solución más extendida y más práctica son las baterías quimicas, como las de plomo-ácido o litio-ion, debido a su capacidad de almacenar grandes cantidades de energía y su vida útil prolongada. En este proyecto, el correcto dimensionamiento y selección de las baterías garantiza un suministro continuo para las operaciones del comercio Huascarán.


  \item \textbf{Dimensionamiento}

  El dimensionamiento de un sistema fotovoltáico aislado implica calcular la capacidad y cantidad de componentes necesarios para satisfacer las necesidades energéticas específicas del usuario. Este proceso incluye el análisis de la demanda energética diaria, la estimación de la radiación solar disponible y la consideración de factores como pérdidas y eficiencia del sistema. Para el comercio Huascarán, el dimensionamiento asegura que el sistema diseñado sea capaz de soportar las cargas críticas durante las operaciones, incluso en condiciones adversas.


  \item \textbf{Cultura de mantenimiento de sistemas fotovoltáicos}

  La cultura de mantenimiento en sistemas fotovoltaicos se refiere a la adopción de prácticas regulares, preventivas y correctivas para asegurar el óptimo desempeño del sistema a lo largo de su vida útil. Este concepto no solo implica realizar intervenciones técnicas específicas, sino también fomentar una mentalidad proactiva en los poseedores del sistema, quienes deben entender la importancia de estas actividades para evitar fallos y maximizar la inversión realizada.

\end{itemize}
