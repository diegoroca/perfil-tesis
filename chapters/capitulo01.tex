En zonas alejadas con infraestructura eléctrica limitada, los cortes frecuentes de electricidad afectan directamente el desempeño y la sostenibilidad de las actividades comerciales. Este problema no solo interrumpe las operaciones diarias, sino que también genera pérdidas económicas y reduce la calidad de los servicios ofrecidos. La dependencia de fuentes de energía no renovables, como los generadores a base de combustible, implica costos elevados y contribuye al impacto ambiental negativo, lo que resalta la necesidad de buscar alternativas más eficientes y sostenibles.
\\
\\
El diseño de un sistema fotovoltaico surge como una solución viable y adaptada a las condiciones específicas de estas regiones. Este tipo de sistema aprovecha la energía solar, una fuente abundante y renovable, para suplir la demanda energética de los comercios y garantizar la continuidad de las operaciones. Sin embargo, para que este enfoque sea efectivo, es necesario un diseño que considere las particularidades del entorno, como el nivel de irradiación solar, los patrones de consumo eléctrico del comercio, y la viabilidad económica a largo plazo. Este trabajo se enfoca en abordar estos desafíos mediante la implementación de un sistema fotovoltaico optimizado para garantizar la autosuficiencia energética en condiciones adversas.


\section{Problema Principal}
¿Cómo diseñar un sistema fotovoltaico que permita garantizar la continuidad del suministro eléctrico en el comercio Huascarán y reducir la dependencia de fuentes no renovables?

\section{Problemas Específicos}
\begin{itemize}
  \item ¿Cómo dimensionar el generador fotovoltaico para satisfacer las necesidades energéticas del comercio?
  \item ¿Cómo implementar un sistema de almacenamiento de energía que permita suplir la demanda eléctrica durante los cortes de electricidad o en horas de baja generación solar?
  \item ¿Cómo evaluar la viabilidad económica del sistema fotovoltaico, incluyendo costos de instalación, mantenimiento y el retorno de inversión esperado?
\end{itemize}

