\section{Internacionales}

\begin{itemize}
  
  \item \textbf{Sistema Solar Fotovoltaico Aislado Para el Suministro de Energía Eléctrica a una vivienda Rural \parencite{Mestre}}

El proyecto plantea como solución el diseño e implementación de un sistema solar fotovoltaico aislado para abastecer de energía eléctrica a una vivienda rural en la zona de Patillal, Cesar, Colombia, caracterizada por su déficit energético y difícil acceso a la red eléctrica convencional. Se evalúan las necesidades energéticas de una vivienda típica, se seleccionan los componentes necesarios del sistema (paneles solares, regulador, inversor, baterías), y se realizan cálculos para dimensionar su capacidad. Concluye que este sistema es económicamente viable al recuperar la inversión en un tiempo razonable.

  \item \textbf{Diseño de un Sistema de Generación Fotovoltaico Residencial Autónomo para el Consumo Nivel 1 \parencite{Dominguez}}

Este trabajo describe el diseño de un sistema de generación fotovoltaico residencial autónomo, orientado a suplir las necesidades de consumo eléctrico de viviendas en áreas rurales de Ecuador, particularmente en San Juan de Ilumán. A partir de la estimación de demanda energética y condiciones locales de radiación solar, se dimensionaron los componentes necesarios: paneles solares, baterías, inversores, reguladores y conductores. El proyecto concluye que, debido a la alta radiación solar constante en Ecuador, esta tecnología es sostenible y eficaz, mejorando la calidad de vida y promoviendo el uso de energía limpia y renovable, a la vez que respalda el cuidado ambiental.

  \item \textbf{Diseño de un sistema aislado para el uso en casas flotantes en la ciudad de Babahoyo \parencite{Puco}}

El documento aborda el diseño de un sistema fotovoltaico aislado para abastecer de energía eléctrica a casas flotantes ubicadas en Babahoyo, Ecuador, en áreas de difícil acceso a la red eléctrica. Con un enfoque experimental, descriptivo y documental, el estudio evalúa la irradiancia local y los componentes requeridos, como paneles solares, baterías, inversores y controladores de carga. Se concluye que esta alternativa es viable, ya que mejora la calidad de vida de los residentes al proporcionar energía renovable y reduce la dependencia de combustibles fósiles, promoviendo la sostenibilidad ambiental.

\end{itemize}


\section{Nacionales}

\begin{itemize}
    
    \item \textbf{Análisis Comparativo de Rendimientos Entre un Sistema Fotovoltaico con Seguidor Solar de Doble Eje y un Distema Fotovoltaico de Montura Fija \parencite{Laureano}}


    \item \textbf{Análisis Costo Beneficio Del Sistema Fotovoltáico Monofásico Conectado a la Red \parencite{Guevara}}


    \item \textbf{Diseño de un Sistema Fotovoltaico con Seguidor Solar para Sistema de Telecomunicación de las Subestaciones Sector Norte \parencite{Correa}}


\end{itemize}
