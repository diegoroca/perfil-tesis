\section{Equipos, Materiales y Servicios}

Para el desarrollo de esta investigación se empleará el siguiente equipo:

\begin{itemize}
  \item Laptop con Procesador Core i3 con 8GB de RAM
  \item Pinza Amperimétrica
  \item Software Matlab
  \item Software PVsyst
  \item Software Helioscope
\end{itemize}

\section{Métodos}
\subsection{Diseño y alcanze de la investigación}

La presente investigación será de tipo aplicada debido a que este proyecto tiene como objetivo desarrollar una solución concreta (el sistema fotovoltaico) para un problema real (los cortes eléctricos que afectan al comercio Huascarán).
\\
\\
Se plantea además un enfoque cuantitativo ya que el diseño del sistema requiere cálculos precisos, como el dimensionamiento de paneles, baterías e inversores, así como la estimación de costos y la evaluación del desempeño energético.
\\
\\
El alcance de esta investigación es describir las condiciones actuales del comercio (problemas energéticos) y las características del sistema propuesto, detallando componentes, funcionamiento y beneficios.
\\
\\
El proyecto no incluye pruebas ni cambios en el sistema eléctrico del comercio Huascarán. El análisis se realiza desde una perspectiva teórica por lo que es un trabajo no experimental.

\subsection{Población y Muestra}

Debido a la naturaleza de la investigación, no se define una población ni una muestra en el sentido convencional. Este trabajo se enfoca exclusivamente en el diseño de un sistema fotovoltaico aislado para el comercio Huascarán, el cual representa un caso particular con características específicas. Si bien los resultados pueden servir como referencia para situaciones similares, el estudio no busca generalizar a un conjunto mayor de establecimientos.

\subsection{Técnicas o métodos a aplicar}

\begin{itemize}

  \item Reecolección de Inromacion documental: Se recopilará y analizará información de fuentes como libros, articulos, normatibas, manuales técncos y bases de datos.
  
  \item Análisis de datos técnicos: Se evaluará los datos específicos, como el consumo energético del comercio, los niveles de radiación solar en la región y las especificaciones técnicas de los equipos disponibles.

  \item Calculos de dimensionamiento: basado en fórmulas y estándares técnicos para determinar la capacidad y cantidad de paneles solares, baterías e inversores necesarios.

  \item Simulacion Energética: Uso de herramientas o software para simular el comportamiento del sistema bajo diferentes condiciones.

  \item Análisis de Viabilidad Técnica y Económica: Evaluar los costos de instalación y operación y mantenimiento, comparándolos con los beneficios energéticos y económicos del sistema para determinar si el sistema es rentable para el comercio Huascarán

  \item Revisión Normativa: Estudio de normativas y estándares nacionales o internacionales relacionados con sistemas fotovoltaicos aislados para asegurar que el diseño cumpla con las regulaciones vigentes, garantizando seguridad y funcionalidad.

\end{itemize}
