Ante la problemática de los cortes frecuentes de energía eléctrica en el distrito de Yanama, el diseño e implementación de un sistema fotovoltaico aislado se justifica como una solución técnica y sostenible. Este sistema garantizará una fuente alternativa de alimentación eléctrica, evitando los paros operativos y asegurando la continuidad de las comunicaciones. De esta manera, se busca no solo mitigar las pérdidas económicas, sino también fortalecer la estabilidad operativa del comercio frente a las limitaciones del suministro energético convencional.
\\
\\
Además de solucionar los problemas los problemas mencionados, el sistema fotovoltaico aislado presenta la posibilidad de sustituir parcialmente el suministro eléctrico convencional durante ciertas jornadas de trabajo. Esto permitiría reducir significativamente los costos asociados al consumo de energía a largo plazo.

\section{Justificación Metodológica}
La metodología empleada en este trabajo se basa en análisis técnico, evaluación de la demanda energética y la selección de componentes óptimos, permitiendo desarrollar una solución eficiente, confiable y adaptada al contexto del comercio.

\section{Justificación Teórica}
El diseño de sistemas fotovoltaicos aislados se fundamenta en principios de conversión de energía solar, eficiencia energética y sostenibilidad. Este trabajo se basa en teorías de generación distribuida y almacenamiento energético, y busca contribuir al conocimiento práctico sobre cómo implementar soluciones autónomas en áreas donde el suministro eléctrico es inestable.

\section{Justificación Práctica}
El desarrollo del sistema fotovoltaico aislado para el comercio Huascarán tendrá un impacto directo en la continuidad de sus operaciones, reduciendo pérdidas económicas por cortes eléctricos. Asimismo, permitirá una reducción de costos operativos mediante el aprovechamiento de energía solar durante jornadas laborales específicas. 
